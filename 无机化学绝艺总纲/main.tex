% !TEX program = xelatex
\documentclass[
  10pt,
  twoside,
  openany,
  b5paper, % 以上均为 ctexbook 提供的文类选项
  colorscheme = basic, % 请根据需要选择或定制配色方案
  xits = false,
]{qyxf-book}



\usepackage{array,tabularx}
\usepackage{subcaption}
\usepackage{ccicons}
\usepackage{draftwatermark}
\usepackage{mhchem}
\usepackage{fontawesome5}
\usepackage{booktabs}
\usepackage{amsmath}
\usepackage{newtxtext}
\usepackage{newtxmath}
\SetWatermarkText{喵小决}
\SetWatermarkLightness{0.92}
\SetWatermarkScale{0.9}

\title{无机化学绝艺总纲}
\subtitle{The Masterpiece of Inorganic Chemistry}  % 可选
\author{喵小决 Kimariyb}
\typo{喵小决 Kimariyb}
\date{2023 年 3 月 4 日}


% 定制元信息
\org{\Large\textit{江西理工大学}\\\textsc{Jiangxi University of Science and Technology}}
\footorg{\textsc{Kimariyb Miao Xiao Jue}}
\cover{\includegraphics[width=.6\textwidth]{logo.pdf}}
\license{}  % 清空许可证信息

% 调整封面标题大小
\renewcommand{\titlefont}{\Huge\bfseries}
\renewcommand{\subtitlefont}{\LARGE\itshape}

\begin{document}

\maketitle

\chapter*{前言}

本作品是一名计算机跨考化学的湖大落榜生编写并排版的,虽然初试成绩不佳,但是还是认真学了复试的内容,并将其归纳整理。本作品大部分参考了明可化学竞赛网站中无机化学部分的总结,删除了一些格林伍德等比较难的无机部分,仅保留比较基础的知识。除此之外,还加上了一些无机常常用到的公式,希望能对大家带来些许帮助。

本作品依旧是使用\LaTeX 进行排版,使用了钱院学府的模板,在这里做出感谢。本作品的初衷是自己使用方便,没有其他想法。

本作品采用\href{https://
	creativecommons.org/licenses/
	by-nc-nd/4.0/}{ BY-
	NC-ND 4.0 协议}进行许可。使用者可以在给出作者署名及资料来源的前提下对本作品进行转载,但不得对本作品进行修改,亦不得基于本作品进行二次创作,不得将本作品运用于商业用途。

由于本人并不是化学专业的科班出身,所以笔误、错漏等在所难免,如您在参考的过程中发现有任何错误之处,欢迎您通过下面的方式联系我们,帮助我们改进这份资料:
\begin{itemize}
	\item \faGithub ~~ GitHub平台论坛:https://github.com/kimariyb/inorganic-chemistry/issues
	\item \faEnvelopeOpen ~~ 本人邮箱:2420707848@qq.com
	\item \faQq ~~ 
	本人QQ:~~\textbf{喵小决}~~2420707848
\end{itemize}

\begin{flushright}
	喵小决\\
	2023 年 3 月 1 日
\end{flushright}

\clearpage
\tableofcontents


\chapter{常用公式}

\section{物理化学、分析化学部分}
\analysis[由于初试中物理化学是作为主要考察对象,因此只列出比较重要的方程。]
\subsection{气体、液体和溶液}

\textbf{理想气体状态方程:}
	\begin{equation*}
		pV = nRT
	\end{equation*}
式中,$R$为气体摩尔常数,通常取$8.314 \ \mathrm{J\cdot mol^{-1}\cdot K^{-1}}$,$n$为物质的量,$T$为热力学温度,$p$为气体压强,$V$为气体体积。该公式主要用来计算理想气体的相关状态。

\textbf{Dalton分压定律:}
	\begin{equation*}
		p_\mathrm{B} =  x_\mathrm{B} p
	\end{equation*}
式中,$x_\mathrm{B}$为气体B的摩尔分数,$p_\mathrm{B}$为气体B的分压,$p$为总压强。

\textbf{Graham扩散定律:}
	\begin{equation*}
		\frac{t_\mathrm{A}}{t_\mathrm{B}} = 	\frac{v_\mathrm{A}}{v_\mathrm{B}} = 	\sqrt{\frac{\rho_\mathrm{A}}{\rho_\mathrm{B}}} = 
		\sqrt{\frac{M_\mathrm{A}}{M_\mathrm{B}}}
	\end{equation*}
式中,$v$称为缝流速率,$M$为摩尔质量,$\rho$为气体密度。

\textbf{van der Waals 方程:}
	\begin{equation*}
		\left(p+\frac{an^2}{V^2}\right)(V-nb) = nRT
	\end{equation*}
式中,$a$、$b$称为van der Waals 常数,$a$用于校正压力,$b$用于校正体积,其数值都是大于零的数。如果没有特殊说明,任何气体都可以当作理想气体。

\textbf{Clausius-Clapeyron 方程:}
	\begin{equation*}
		\ln \frac{p_2}{p_1} = \frac{\Delta_\mathrm{vap}H_\mathrm{m}}{R} \left(\frac{1}{T_1}-\frac{1}{T_2}\right)
	\end{equation*}
式中,$\Delta_\mathrm{vap}H_\mathrm{m}$是摩尔气化焓。用来计算不同温度下的饱和蒸气压。

\textbf{Raoult 定律:}
	\begin{equation*}
		p_i = p^\ast_i x_i
	\end{equation*}
式中,$p^\ast_i$是纯溶剂的蒸气压,$p_1$是蒸气压下降之后的蒸气压。只有难挥发的、非电解质溶液的蒸气压才能使用Raoult定律进行定量计算。其他的溶液虽然不满足Raoult定律,但依旧会造成蒸气压下降。

\textbf{Henry 定律:}
	\begin{equation*}
		k_\mathrm{H} = \frac{p}{x}
	\end{equation*}
式中,$k_\mathrm{H}$为Henry常数,$p$是被溶解气体的分压,$x$是溶解气体在溶液中的物质的量分数。需要特别注意Henry常数的单位,并且Henry定律只适用于稀溶液。

\textbf{沸点升高公式:}
	\begin{equation*}
		\Delta T_\mathrm{b} = k_\mathrm{b} b 
	\end{equation*}
式中,$k_\mathrm{b}$是溶剂的沸点升高常数,$b$为溶质的质量摩尔浓度。

\textbf{凝固点降低公式:}
	\begin{equation*}
		\Delta T_\mathrm{f} = k_\mathrm{f} b 
	\end{equation*}
式中,$k_\mathrm{f}$是溶剂的凝固点降低常数,$b$为溶质的质量摩尔浓度。

\textbf{渗透压的计算:}
	\begin{equation*}
		\varPi V = nRT
	\end{equation*}
式中,$\varPi$为渗透压。上述公式只适用于非电解质稀溶液,只有在半透膜的存在下,才能表现出渗透压。

\subsection{化学热力学}


\textbf{热力学第一定律:}
	\begin{equation*}
		\Delta U = Q + W 
	\end{equation*}
式中,$\Delta U$表示热力学能的改变量,$W$表示功,$Q$表示热。

\textbf{化学反应中的焓变:}
	\begin{equation*}
		\Delta H = \Delta U + \Delta n(\mathrm{g})RT
	\end{equation*}
式中,$\Delta H$表示焓变。其定义为$H=U+PV$。

\textbf{生成焓计算反应焓:}
	\begin{equation*}
		\Delta_\mathrm{r}H^\ominus_\mathrm{m} = \sum \Delta_\mathrm{f}H^\ominus_\mathrm{m}(\text{生成物}) - \sum \Delta_\mathrm{f}H^\ominus_\mathrm{m}(\text{反应物}) 
	\end{equation*}
式中,$\Delta_\mathrm{f}H^\ominus_\mathrm{m}$表示标准摩尔生成焓。其定义为在指定温度和标准压力下,由最稳定单质生成1 mol物种的反应热。

\textbf{燃烧焓计算反应焓:}
	\begin{equation*}
		\Delta_\mathrm{r}H^\ominus_\mathrm{m} = \sum 	\Delta_\mathrm{c}H^\ominus_\mathrm{m}(\text{反应物}) - \sum \Delta_\mathrm{c}H^\ominus_\mathrm{m}(\text{生成物}) 
	\end{equation*}
式中,$\Delta_\mathrm{c}H^\ominus_\mathrm{m}$表示标准摩尔燃烧焓。其定义为在指定温度和标准压力下,1 mol有机物完全燃烧时发生的热量变化。完全燃烧是指有机物中的$\ce{C}$变为$\ce{CO2(g)}$,$\ce{H}$变为$\ce{H2O(l)}$,$\ce{S}$变为$\ce{SO2(g)}$,$\ce{N}$变为$\ce{N2(g)}$,$\ce{Cl}$变为$\ce{HCl(aq)}$。

\textbf{标准熵计算反应熵:}
	\begin{equation*}
		\Delta_\mathrm{r}S^\ominus_\mathrm{m} = \sum  S^\ominus_\mathrm{m}(\text{生成物}) - \sum  S^\ominus_\mathrm{m}(\text{反应物}) 
	\end{equation*}
式中,$S^\ominus_\mathrm{m}$称为标准熵,单质的标准熵不等于零,温度对化学反应熵变的影响不大,因此有时可以忽略。

\textbf{Gibbs-Helmholtz方程式:}
	\begin{equation*}
		\Delta G = \Delta H -T\Delta S
	\end{equation*}
式中,$\Delta G$是Gibbs 自由能,在恒温、恒压下可以用来判断反应的进行方向,使用时需要注意温度是否匹配。该方程可以用来求转向温度、判断反应方向等。

\textbf{生成Gibbs自由能计算反应Gibbs自由能:}
	\begin{equation*}
		\Delta_\mathrm{r}G^\ominus_\mathrm{m} = \sum \Delta_\mathrm{f}G^\ominus_\mathrm{m}(\text{生成物}) - \sum \Delta_\mathrm{f}G^\ominus_\mathrm{m}(\text{反应物}) 
	\end{equation*}
式中,$\Delta_\mathrm{f}G^\ominus_\mathrm{m}$称为标准Gibbs生成自由能,它定义为在标态和指定温度下,由稳定单质生成1 mol化合物的Gibbs自由能变。

\textbf{van't Hoff 等温式:}
	\begin{equation*}
		\Delta_\mathrm{r} G_\mathrm{m} = \Delta_\mathrm{r} G^\ominus_\mathrm{m} + RT\ln Q
	\end{equation*}
式中,$Q$称为反应商,其量纲为1。该等温式可以用来计算非标准状态下的自由能变化。对于任一化学变化:
\begin{equation*}
	\ce{aA(aq) + bB(l) -> dD(g) + eE(s)}
\end{equation*}
其反应商$Q$为:
\begin{equation*}
	Q = \frac{(p_\mathrm{D}/p^\ominus)^d}{(c_\mathrm{A}/c^\ominus)^a}
\end{equation*}

\textbf{标准平衡常数与Gibbs自由能之间的关系:}
	\begin{equation*}
		\Delta_\mathrm{r} G^\ominus_\mathrm{m} = -RT \ln K^\ominus
	\end{equation*}
如果反应此时没有达到平衡状态,则:
	\begin{equation*}
		\Delta_\mathrm{r} G_\mathrm{m} = -RT \ln K^\ominus + RT \ln Q
	\end{equation*}
式中,$K^\ominus$为标准平衡常数,其量纲为1,在计算时必须采用相对压力或相对浓度。

\textbf{van't Hoff 方程:}
	\begin{equation*}
		\ln \frac{K^\ominus_2}{K^\ominus_1} = \frac{\Delta_\mathrm{r} H^\ominus_\mathrm{m}}{R} \left(\frac{1}{T_1}- \frac{1}{T_2}\right)
	\end{equation*}
van't Hoff 方程式可以求解不同温度下的标准平衡常数,也可以根据标准平衡常数求解反应焓。使用这个方程的前提是,默认反应焓随温度的影响不大。

\textbf{一元弱酸和一元弱碱的近似公式:}
	\begin{equation*}
		[\ce{H+}] = -\frac{K_\mathrm{a}}{2}+\frac{K_\mathrm{a}}{2}\sqrt{1+\frac{4c_0}{K_\mathrm{a}}} \approx \sqrt{K_\mathrm{a}c_0}
	\end{equation*}
式中,$K_a$为一元弱酸的解离常数,要想近似成立,其必要条件为$c_0/K_a > 400$。

\textbf{Henderson-Hasselbach 方程式:}
	\begin{equation*}
		\mathrm{pH}  = \mathrm{p}K_\mathrm{a} + \lg \frac{[\ce{A-}]}{[\ce{HA}]}
	\end{equation*}
该方程式,可以用来计算一对缓冲溶液所提供的pH值。

\textbf{缓冲溶液的有效缓冲范围:}
	\begin{equation*}
		\mathrm{pH} = \mathrm{p}K_\mathrm{a} \pm 1
	\end{equation*}

任何一个缓冲溶液都有一个有效范围,如果超出这个量,缓冲溶液就会失去平衡而失效。

\textbf{多元弱酸的电离规律:}

(1)多元弱酸的氢离子浓度是按第一级电离平衡来计算的;

(2)二元弱酸的酸根离子浓度等于第二级电离常数;
 
(3)饱和\ce{H2S}的物质的量浓度为1.0 $\mathrm{mol \cdot dm^\mathrm{-3}}$。

\textbf{弱碱弱酸盐水解的近似公式:}
	\begin{equation*}
		[\ce{H+}] \approx \sqrt{\frac{K_\mathrm{a}\overline{K_\mathrm{a}}\cdot c_0}{K_\mathrm{a}+c_0}} \approx \sqrt{K_\mathrm{a} \cdot \overline{K_\mathrm{a}}}
	\end{equation*}
式中,$\overline{K_\mathrm{a}}$称为共轭酸的电离常数。第一个近似要想成立,则需要满足$c_0 \cdot \overline{K_\mathrm{a}} \gg K_\mathrm{W}$。第二个近似要想成立,需要在满足第一个近似成立的条件下,还需满足$c_0 \gg K_\mathrm{a}$。

\subsection{电化学}

\textbf{电极电势与Gibbs自由能的关系:}
	\begin{equation*}
		\Delta_\mathrm{r}G^\ominus_\mathrm{m} = -nF\varepsilon^\ominus
	\end{equation*}
式中,$F$为法拉第常数,值为96500 C;$\varepsilon^\ominus$为标准电池电势。

\textbf{Nernst 方程:}
	\begin{equation*}
		\varepsilon = \varepsilon^\ominus - \frac{RT}{nF} \ln Q
	\end{equation*}
Nernst 方程还有一种形式,对于$[\ce{H+}] = 1 \ \mathrm{mol \cdot dm^\mathrm{-3}}$,$p = p^\ominus$,则Nernst方程可以修改为:
	\begin{equation*}
		\varphi_\mathrm{Ox/Red}=\varphi^\ominus_\mathrm{Ox/Red} - \frac{0.0592}{n} \lg \frac{[\mathrm{Red}]}{[\mathrm{Ox}]}
	\end{equation*}

使用电极的nernst方程需要注意的是[Ox],[Red]项要乘以与电极反应中Ox和Red物种前面系数相同的次方。如果电对中的某一物质是固体或液体,则它们的浓度均为常数,常认为是1。电对中的某物质是气态,则要用以大气压为单位的气体分压数据。

\subsection{化学动力学}

\textbf{化学速率的表示:} 对于一个气相反应:$\ce{2NO2(g) + F2(g) -> 2NO2F(g)}$,其反应速率表示为:
	\begin{equation*}
		\mathrm{rate} = -\frac{1}{2} \frac{\mathrm{d}[\ce{NO2}]}{\mathrm{d}t} =  - \frac{\mathrm{d}[\ce{F2}]}{\mathrm{d}t} =  \frac{1}{2} \frac{\mathrm{d}[\ce{NO2F}]}{\mathrm{d}t}
	\end{equation*}

\textbf{质量作用定律:}
	\begin{equation*}
		\mathrm{rate} = k[\mathrm{A}]^x[\mathrm{B}]^y
	\end{equation*}
仅适用于基元反应。式中,$k$称为速率常数,只和反应的性质和温度有关。

\textbf{零级反应的反应速率方程:}
	\begin{equation*}
		[\mathrm{A}]_t = [\mathrm{A}]_0 -k_0t
	\end{equation*}

\textbf{一级反应的速率方程:}
	\begin{equation*}
		\ln \frac{[\mathrm{B}]_0}{[\mathrm{B}]_t} = k_1t
	\end{equation*}

\textbf{半衰期法求解反应级数:}
	\begin{equation*}
		n = 1-\frac{\ln \frac{t^\prime_{1/2}}{t_{1/2}}}{\ln \frac{[\mathrm{A}]^\prime_0}{[\mathrm{A}]_0}}
	\end{equation*}

\textbf{Arrhenius 公式:}
	\begin{equation*}
		k = A \ \mathrm{exp}\left(-\frac{E_\mathrm{a}}{RT}\right)
	\end{equation*}
式中,$A$称为指前因子,$E_\mathrm{a}$为活化能,Arrhenius 公式还有另一种形式:
	\begin{equation*}
		\ln \frac{k_2}{k_1} = \frac{E_\mathrm{a}}{R} \left(\frac{1}{T_1}-\frac{1}{T_2}\right)
	\end{equation*}
该式可以用来计算不同温度的反应速率常数,或者已知反应速率常数求解活化能。

\newpage

\section{结构化学部分}

\subsection{原子、分子结构}

\textbf{Slater 规则:}
	\begin{equation*}
		E = - \frac{(Z-\sigma)^2}{n^2} \times 13.6 \ \mathrm{(eV)}
	\end{equation*}
Slater 规则中的屏蔽因子的计算规则:

(1)原子中的电子分在若干个轨道组中:(1s),(2s,2p),(3s,3p),(3d)等,每个阔号形成一个轨道组。

(2)一个轨道组外面的轨道组上的电子对内轨道的电子的屏蔽系数$\sigma = 0$。这说明屏蔽作用仅发生在内层电子对外层电子或同层电子之间。

(3)同一轨道组内电子间屏蔽系数$\sigma=0.35$,1s轨道上的2个电子之间的屏蔽系数$\sigma=0.30$。

(4)被屏蔽电子为$n\mathrm{s}$或$n\mathrm{p}$轨道组上的电子时,主量子数为$(n-1)$的各轨道上的电子,对$n\mathrm{s}$或$n\mathrm{p}$轨道组上的电子的屏蔽系数$\sigma=0.85$,而小于$(n-1)$的各轨道上的电子,对其屏蔽系数$\sigma=1.00$。

(5)被屏蔽电子为$n\mathrm{d}$或$n\mathrm{f}$轨道组上的电子时,则位于它左边各轨道组上的电子,对$n\mathrm{d}$或$n\mathrm{f}$轨道组上电子的屏蔽系数$\sigma=1.00$。

\textbf{分子轨道键级的计算:}
	\begin{equation*}
		\text{键级} =  \frac{1}{2}\times(\text{成键轨道电子数}-\text{反键轨道电子数})
	\end{equation*}
$\ce{NO}$、$\ce{NO+}$和$\ce{NO-}$的键级分别为2.5、3和2。

\subsection{配位化学}

\textbf{磁矩的计算:}
	\begin{equation*}
		\mu_\mathrm{s} = \sqrt{n\cdot(n+2)} \ \mathrm{B.M.}
	\end{equation*}

\textbf{八面体场中分裂能的计算:}
	\begin{equation*}
		\Delta_\mathrm{o} = E(\mathrm{e_g}) - E(\mathrm{t_{2g}})
	\end{equation*}

\textbf{八面体场中CFSE的计算:}
	\begin{equation*}
		(\mathrm{CFSE})_\mathrm{o} = (-4D\mathrm{q}) \times n(\mathrm{t_{2g}}) + 6D\mathrm{q} \times n(\mathrm{e_{g}})
	\end{equation*}

\textbf{四面体场中分裂能的计算:}
\begin{equation*}
	\Delta_\mathrm{t} = E(\mathrm{t_{2g}}) - E(\mathrm{e_{g}})
\end{equation*}

\textbf{四面体场中CFSE的计算:}
\begin{equation*}
	(\mathrm{CFSE})_\mathrm{t} = (-2.67D\mathrm{q}) \times n(\mathrm{e_{g}}) + 1.78D\mathrm{q} \times n(\mathrm{t_{2g}})
\end{equation*}

\chapter{元素化学}

\section{元素化学通论}

\subsection{含氧酸强度}

\textbf{\ce{R-O-H}规则} \quad 含氧酸在水溶液中的强度决定于酸分子中质子转移倾向的强弱,质子转移倾向越大,酸性
越强,反之则越弱。而质子转移倾向的难易程度,又取决于酸分子中 R 吸引羟基氧原子的电子的能力,
当 R 的半径较小,电负性越大,氧化数越高时,R 吸引羟基氧原子的能力强,能够有效的降低氧原子
上的电子密度,使 O-H 键变弱,容易放出质子,表现出较强的酸性。

\textbf{Pauling 规则} \quad 含氧酸的通式是 $\ce{RO_n(OH)_m}$,$n$为非氢键合的氧原子数(非羟基氧),$n$值越大酸性越强,
并根据$n$值把含氧酸分为弱酸($n=0$),中强酸($n=1$),强酸($n=2$),极强酸($n=3$)四类。因为酸分
子中非羟基氧原子数越大,表示分子中$\ce{R-O}$配键越多,R 的还原性越强,多羟基中氧原子的电子吸
引作用越大,使氧原子上的电子密度减小的越多,$\ce{O-H}$键越弱,酸性也就越强。注意:应用此规则
时,只能使用结构式判断,而不能使用最简式。

含氧酸脱水“缩合”后,酸分子内的非氢键合的氧原子数会增加,导致其酸性增强,多酸的酸性比原来
的酸性强。

(1)同一周期,同种类型的含氧酸,其酸性自左向右依次增强:

如:$\ce{HClO4} > \ce{H2SO4} > \ce{H3PO4} > \ce{H4SiO4} $

(2)同一族中同种类型的含氧酸,其酸性自上而下依次减弱:

如:$\ce{HClO} > \ce{HBrO} > \ce{HIO}  $

(3)同一元素不同氧化态的含氧酸,高氧化态含氧酸的酸性较强,低氧化态含氧酸的酸性较弱:

如:$\ce{HClO4} > \ce{HClO3} > \ce{HClO2} > \ce{HClO} $
 
\analysis[注意:也有例外情况,例如酸性$\ce{H3PO2} > \ce{H3PO3} > \ce{H3PO4}$。]

\subsection{含氧酸稳定性与氧化性}

(1)同一元素的含氧酸,高氧化态的酸比低氧化态的酸稳定:

如:$\ce{HClO4} > \ce{HClO3} > \ce{HClO2} > \ce{HClO} $

(2)同一周期主族元素和过渡元素最高价含氧酸氧化性随原子序数递增而增强:

如:$\ce{BrO4^-}>\ce{MnO4^-}$,$\ce{V2O5} < \ce{Cr2O7^2-}$

(3)同一元素不同氧化态的含氧酸中,低氧化态的氧化性较强:

如:$\ce{HClO} > \ce{HClO2} > \ce{HClO3} $

(4)同一主族中,各元素的最高氧化态含氧酸的氧化性,大多随原子序数增加呈锯齿形升高:

如:$\ce{HNO3} > \ce{H3PO4} < \ce{H3AsO4}$ ,$\ce{H2SO4} < \ce{H2SeO4} > \ce{H6TeO6}$ 

低氧化态则自上而下有规律递减:

如:$\ce{HClO} > \ce{HBrO} > \ce{HIO} $

5)浓酸的含氧酸氧化性比稀酸强,含氧酸的氧化性一般比相应盐的氧化性强,同一种含氧酸盐在酸性
介质中比在碱性介质中氧化性强。

\begin{note}
	\textbf{影响含氧酸(盐)氧化能力的因素:}
	
	\begin{itemize}
		\item 中心原子结合电子的能力:若中心原子半径小,电负性大,获得电子的能力强,其含氧酸(盐)的
		氧化性也就强,反之,氧化性则弱。高氧化态氧化性锯齿形变化则是由于次级周期性引起的。
		\item 含氧酸分子的稳定性:含氧酸的氧化态和分子的稳定性有关,一般来说,如果含氧酸分子中的中心
		原子R多变价,分子又不稳定,其氧化性越强。稳定的多变价元素的含氧酸氧化性很弱,甚至没有氧
		化性。低氧化态含氧酸氧化性强还和它的酸性弱有关,因为在弱酸分子中存在着带正电性的氢原子,对酸分子中的R原子有反极化作用,使$\ce{R-O}$键易于断裂。
		\item 同理可以解释:(1)为什么浓酸的氧化性比
		稀酸强?因为在浓酸溶液中存在着自由的酸分子,有反极化作用。(2)为什么含氧酸的氧化性比含氧酸
		盐强?因为含氧酸盐中$\ce{ M^n+}$反极化作用比$\ce{H+}$弱,含氧酸盐比含氧酸稳定。
	\end{itemize}
\end{note}

\subsection{含氧酸盐的热稳定性规律}

(1)同一盐及其酸稳定性次序是:正盐$>$酸式盐$>$酸:

如:$\ce{Na2CO3}>\ce{NaHCO3}>\ce{H2CO3}$

(2)同一酸根不同金属的含氧酸盐,热稳定性次序是:碱金属$>$碱土金属$>$过渡金属$>$铵盐:

如:$\ce{K2CO3}>\ce{CaCO3}>\ce{ZnCO3}>\ce{(NH4)2CO3}$

(3)同一酸根同族金属离子盐,热稳定性从上到下一次递增:

如:$\ce{BeCO3}>\ce{MgCO3}>\ce{CaCO3}>\ce{SrCO3}>\ce{BaCO3}$

(4)同一成酸元素其高氧化态含氧酸盐比低价态稳定:

如:$\ce{KClO4}>\ce{KClO3}>\ce{KClO2}>\ce{KClO}$

(5)不同价态的同一金属离子的含氧酸盐,其低价态比高价态稳定:

如:$\ce{Hg2(NO3)2}>\ce{Hg(NO3)2}$

(6)酸不稳定,其盐也不稳定,酸越稳定,其盐也较稳定。碳酸盐,硝酸盐,亚硫酸盐,卤酸盐的稳定性都
较差,较易分解;硫酸盐,磷酸盐较稳定,其酸也较稳定,难分解。这是由于金属离子的反极化作用
越大,该盐的热稳定性就越差。

如:$ \ce{Na3PO4}>\ce{Na2SO4}>\ce{Na2CO3}>\ce{NaNO3}$


\subsection{p 区元素的次级周期性}

次级周期性是指元素周期表中,每族元素的物理化学性质,从上向下并非单调的直线式递变,而是呈
现起伏的“锯齿形”变化,对于 p 区元素,主要是指第二,第四,第六周期元素的正氧化态,尤其是
最高氧化态的化合物所表现的特殊性。
	
\textbf{第二周期p区元素的特殊性}

1)N、O、F 的含氢化合物容易形成氢键,离子性较强。

2)它们的最高配位数为4,而第3周期和以后几个周期的元素可以超过4。

3)与同族元素相比,除稀有气体外,B、C、N、O、F 内层电子少,只有 $\mathrm{1s^2}$,原子半径特别小(同一
族中,从第二周期到第三周期原子半径增加幅度最大),价轨道没有 d 轨道等特点,所以第二周期元
素的电子亲和能$A$反常地比第三周期同族元素的小。

\textbf{第四周期 p 区元素的不规则性}

最突出的反常性质是最高氧化态化合物(如氧化物,含氧酸及其盐)的稳定性小,而氧化性则很强。
第四周期 p 区元素,经过 d 区长周期中的元素,内层增加了 10 个 d 电子,次外层结构是 $\mathrm{3s^2 3p^6 3d^{10}}$,
由于 d 电子屏蔽核电荷能力比同层的 s、p 电子的要小,这就使从 Ga到Br,最外层电子感受到有效核
电荷$Z^\ast$比不插入 10 个 d 电子时要大,导致这些元素的原子半径和第三周期同族元素相比,增加幅度
不大。由原子半径引起的这些元素的金属性(非金属性)、电负性、氢氧化物酸碱性、最高氧化态含氧
酸(盐)的氧化性等性质都出现反常现象,即所谓“不规则性”。

导致第四周期 p 区元素性质不规则性的本质因素是因为第三周期过渡到第四周期,次外层电子从 $\mathrm{2s^2 2p^6}$
变为 $\mathrm{3s^2 3p^6 3d^{10}}$,第一次出现了 d 电子,导致有效核电荷 $Z^\ast$增加得多,使最外层的 4s 电子能级变低,
比较稳定。

\textbf{p 区金属$ \mathrm{6S^2}$电子的稳定性}

周期表中 p 区下方的金属元素,即第六周期的 Tl,Pb,Bi在化合物中的特征氧化态应依次为+III,
+ IV和+V,但这三种元素的氧化态表现反常,它们的低氧化态化合物,既 Tl(I),Pb(II),Bi(III),
的化合物最稳定。这种现
象称之为“惰性电子对效应”。

产生惰性电子对效应,原因是多方面的,仅从结构上考虑主要有:从第四周期过渡到第五周期,原子
的次外层结构相同,所以同族元素相应的化合物性质改变较有规律。从第五到第六周期,次外电子层
虽相同,但倒数第三层电子结构发生改变,第一次出现了 4f 电子,由于 f 电子对核电荷的屏蔽作用比
d 电子更小,以使有效电荷$Z^\ast$也增加得多,$\ce{6s^2}$也变得稳定,所以第六周期 p 区元素和第五周期元素相
比,又表现出一些特殊性。

\subsection{无机化合物的水解规律}

(1)随正,负离子极化作用的增强,水解反应加剧,这包括水解度的增大和水解反应的深化。离子电
荷,有效核电荷,离子半径是影响离子极化作用强弱的主要内在因素,电
荷高,半径小的离子,其极化作用强。由 18 电子(如 $\ce{Cu+}$,$\ce{Hg^2+}$等),18+2电子(如$\ce{Sn^2+}$,$\ce{Bi^3+}$)
以及 2 电子($\ce{Li+}$,$\ce{Be^2+}$)的构型过度到 9-17 电子(如 $\ce{Fe^3+}$,$\ce{Co^2+}$)构型,8 电子构型时,离子极
化作用依次减弱。共价型化合物水解的必要条件是电正性原子要有空轨道。

(2)温度对水解反应的影响较大,是主要的外因,温度升高时水解加剧。

(3)水解产物不外乎碱式盐,氢氧化物,含水氧化物和酸四种,这个产物顺序与正离子的极化作用增
强顺序是一致的。低价金属离子水解的产物一般为碱式盐,高价金属离子水解的产物一般为氢氧
化物或含水氧化物。在估计共价型化合物的水解产物时,首先要判断清楚元素的正负氧化态,判
断依据就是它们的电负性。在 P,S,Br,Cl,N,F 这系列中,元素在相互化合时,处于右位的
为负性。负氧化态的非金属的水解产物一般为氢化物,正氧化态的非金属元素的水解产物一般为
含氧酸。

\subsection{无机物的酸解反应}

\begin{table}[htbp]
	\centering
	\begin{tabular}{c}
	\toprule
		反应方程式  \\
	\midrule
		\qquad \qquad $\ce{SO3^2- + 2H+ -> SO2 ^ + H2O}$ \qquad\qquad \\
		\qquad\qquad $\ce{3NO_2- + 2H+ -> NO3^- + 2NO ^ + H2O}$ \qquad\qquad\\
		\qquad\qquad $\ce{S2O3^2- + 2H+ -> SO2 ^ + S v + H2O}$ \qquad\qquad \\  \qquad\qquad $\ce{S2^2- + 2H+ -> S v + H2S ^}$ \qquad\qquad \\
		\qquad\qquad $\ce{S_x^2- + 2H+ -> H2S ^ + (x-1)S v }$\qquad\qquad \\  \qquad\qquad$\ce{SnS3^2- + 2H+ -> SnS2 v + H2S ^}$ \qquad\qquad \\
		\qquad\qquad $\ce{2AsS3^3- + 6H+ -> As2S3 v + 3H2S ^ }$ \qquad\qquad \\   \qquad\qquad $\ce{2AsS4^3- + 6H+ -> As2S5 v + 3H2S ^}$ \qquad\qquad \\
		\qquad\qquad $\ce{2SbS4^3- + 6H+ -> Sb2S5 v + 3H2S ^}$\qquad\qquad\\
		\qquad\qquad $\ce{Mg2Si + 4H+ -> 2Mg^2+ + SiH4}$ \qquad\qquad \\
		\qquad\qquad $\ce{Fe2S3 + 4H+ -> 2Fe^2+ + S v + H2S ^}$\qquad\qquad\\
	\bottomrule
	\end{tabular}
\end{table}

\subsection{非金属单质的碱歧化反应}
\begin{table}[htbp]
	\centering
	\begin{tabular}{c}
		\toprule
		反应方程式  \\
		\midrule
		\qquad \qquad $\ce{Cl2 + 2KOH -> KCl + KClO + H2O}$ \qquad\qquad \\
		\qquad\qquad $\ce{2F2 + 2OH- -> OF2 + H2O + 2F-}$ \qquad\qquad\\
		\qquad\qquad $\ce{3I2 + 6OH- -> 5I- + IO3^- + 3H2O}$ \qquad\qquad \\  \qquad\qquad $\ce{3S + 6NaOH -> 2Na2S + Na2CO3 + 3H2O}$ \qquad\qquad \\
		\qquad\qquad $\ce{4P + 3NaOH + 3H2O -> 3NaH2PO2 + PH3 ^}$\qquad\qquad \\  \qquad\qquad$\ce{Si + 2OH- + H2O -> SiO3^2- + 2H2 ^}$ \qquad\qquad \\
		\qquad\qquad $\ce{2B + 2NaOH + 3KNO3 -> 2NaBO2 + 3KNO2 + H2O}$ \qquad\qquad \\   
		\bottomrule
	\end{tabular}
\end{table}

\subsection{含氧酸盐热分解的自身氧化还原规律}

含氧酸盐受热分解,如果有电子转移,而且这种转移是在含氧酸盐内部进行的话,就发生自身氧化还原反应

如:$\ce{2AgNO3 ->[{\mbox{加热}}]2Ag + 2NO2 ^ + O2 ^}$

(1)阴离子氧化阳离子反应:阴离子具有较强氧化性而阳离子又有较强还原性,如$\ce{NH4NO3}$,  $\ce{(NH4)2Cr2O7}$等。

$\ce{NH4NO2 ->[{\mbox{加热}}]N2 ^ + 2H2O}$ \qquad $\ce{(NH4)2Cr2O7 ->[{\mbox{加热}}]Cr2O3 + N2 ^ + 4H2O}$

(2)阳离子氧化阴离子的反应:如果含氧酸盐中阳离子具有强氧化性,而阴离子具有强的还原性,则受热后可能在阴阳离子之间发生
氧化还原反应,如:

$\ce{AgNO2 ->[{\mbox{加热}}]Ag + NO2 ^}$ \qquad \quad $\ce{Ag2SO3 ->[{\mbox{加热}}]2Ag + SO3 ^}$ 

又如:$\ce{HgSO4 ->[{\mbox{加热}}]Hg + O2 ^ + SO2 ^}$

在盐热分解较多见主要是 Ag 和 Hg 的含氧酸盐易发生这种反应。

(3)阴离子自身氧化还原反应:如果含氧酸盐中阳离子稳定,阴离子不稳定($\ce{ClO_4-}$、$\ce{NO_3-}$、$\ce{MnO_4-}$),而且相应的酸性氧化物($\ce{Cl2O7}$、
$\ce{N2O5}$、$\ce{Mn2O7}$)也不稳定时,则它们受热时,只能在阴离子内部不同元素之间发生电子的转移而使化
合物分解,通常为阴离子自身氧化还原反应,分解时,通常有氧气放出。

\begin{note}
	硝酸盐受热分解:热分解产物因金属离子的性质不同而分为如下三类:
	
	(1)最活泼的金属(比Mg活泼的金属)的硝酸盐受热分解产生亚硝酸盐和氧气:$\ce{2NaNO3 ->[{\mbox{加热}}]2NaNO2 + O2 ^}$
	
	(2)活泼性较差的金属(活泼性位于 Mg 和 Cu 之间的金属)的硝酸盐受热分解为氧气、二氧化氮和相应
	的金属氧化物:$\ce{2Pb(NO3)2 ->[{\mbox{加热}}]2PbO + 4NO2 ^ + O2 ^}$
	
	(3)不活泼金属(比 Cu 更不活泼的金属)的硝酸盐受热分解为氧气,二氧化氮和金属单质:$\ce{2AgNO3 ->[{\mbox{加热}}]2Ag + 2NO2 ^ + O2 ^}$

\end{note}

\subsection{含氧酸盐热分解的歧化反应规律}

这种类型热分解虽也属氧化还原反应,但其氧化还原反应是发生在同一元素上,结果使该元素的氧化数一
部分变高,另一部分则变低,如 $\ce{NaClO}$、$\ce{Na2SO3}$、$\ce{Cu2SO4}$等。

(1)阴离子的歧化反应需要具备如下三个条件:

1、成酸元素的氧化态处于中间价态;

2、酸根阴离子必须是不稳定的,而且歧化后元素的价态是稳定的,例如 $\ce{ClO3^-}$可歧化为 $\ce{Cl-}$和 $\ce{ClO4-}$;

3、含氧酸盐中阳离子必须稳定,它们都是碱金属和少数活泼的碱土金属离子等,如:

$\ce{3NaClO ->[{\mbox{加热}}]2NaCl + NaClO3}$ \qquad $\ce{4Na2SO3 ->[{\mbox{加热}}]Na2S + 3Na2SO4}$

应注意这三个条件必须同时具备,否则不发生这类反应,如亚硝酸钾$ \ce{KNO2}$和亚硝酸银$ \ce{AgNO2}$中,氮原
子处中间价态,但由于硝酸根不如亚硝酸根稳定,因此受热时不会发生这种类型的反应。

(2)阳离子歧化反应:含氧酸盐中,若阳离子不稳定时,加热也可能发生歧化分解,如:

$\ce{Hg2CO3 ->[{\mbox{加热}}]HgO + Hg + CO2}$ \qquad $\ce{Mn2(SO4)3 ->[{\mbox{加热}}]MnO2 + MnSO4 + 2SO3}$

综上所述:在常见的含氧酸盐中磷酸盐、硼酸盐、硅酸盐都比较稳定,它们在加热时不分解,但易脱水缩
合为多酸盐;硝酸盐及卤酸盐不稳定,由于它们的酸根离子具有氧化性,因此加热这类盐会发生不同
形式的氧化还原反应,随金属阳离子的不同产物各异,如硝酸盐的几种类型;碳酸盐和硫酸盐等居中,
且硫酸盐的分解温度高于碳酸盐,一般含氧酸盐的酸式盐不如正盐稳定。

\subsection{金属元素高低价转化的规律}

同一金属的多种不同价态在溶液中存在的形式
不同,它们都以各自的最稳定状态而存在于溶液中。处于低价态的金属离子一般以简单的阳离子形
式存在于溶液中,如 $\ce{Pb^2+}$、$\ce{Mn^2+}$、$\ce{Fe^2+}$、$\ce{Bi^3+}$、$\ce{Cr^3+}$等;处于中间价态的金属元素大都以氧化物、氧
酰离子或相应价态的酸根离子形式存在于溶液中,如 $\ce{MnO2}$、$\ce{PbO+}$、$\ce{CrO2-}$等;处于高价态的金属元
素常以复杂的含氧酸根形式存在,如 $\ce{MnO4-}$、$\ce{BiO3-}$、$\ce{CrO4^2-}$、$\ce{FeO4^2-}$等;这主要是因为同一金属元素
离子价态越高,半径就越小,离子电荷与半径的比值越大的离子,对水分子的极化力大。由于极化,
使 $\ce{O-H}$ 键电子密度减少,易断键,结果,由水配位的金属离子转化为羟基配位的金属离子,进一步
转化为氧配位的配合阴离子如$\ce{MnO4-}$。

(1)由低价态化合物转化到高价态化合物,需在碱性介质中用氧化剂氧化低价态的离子,如:

$\ce{Fe^2+ ->Fe(OH)2 ->Fe(OH)3 ->FeO4^2-}$

在碱性介质中能完成这种转化过程,有几点原因:从电极电势来看在碱性介质中金属的电对较低,
其还原态不稳定,还原性较强,易找合适的氧化剂将低价态氧化成高价态。这些元素的高价态在酸
性介质中极不稳定,只能在碱性中存在。当然也
有一些高价态在酸中稳定,不一定用碱,但只是少数,如:

$\ce{Sn^2+ ->Sn^4+}$ \qquad $\ce{Sn^2+ ->Sn^4+}$ \qquad $\ce{Ce^3+ ->Ce^4+}$

绝大多数金属由低价态转化到高价态需在碱性介质中进行,是主要的制备
原则(碱性介质加上强氧化剂是制备高价态的一个原则)。

(2)由高价态向低价态转化需在酸性介质条件下,用强还原剂将高价化合物还原,如:

$\ce{PbO2 ->Pb^2+}$ \qquad $\ce{NaBiO3 ->Bi^3+}$ \qquad $\ce{CrO4^2- ->Cr^3+}$

在酸性介质中,电对的值增大,其高价态的氧化性增强,在强还原剂的作用下可以转化为低价态的金
属离子。
\section{单质的制取}

\subsection{热分解法}

在金属活动顺序中,在氢后面的金属其氧化物受热就容易分解,如$\ce{HgO}$ 和$ \ce{Ag2O}$ 加热发生下列分解反应:

$\ce{HgO ->[{\mbox{加热}}]2Hg + O2 ^}$ \qquad $\ce{2Ag2O ->[{\mbox{加热}}]4Ag + O2 ^}$

\subsection{热还原法}
大量的冶金过程属于这种方法。焦炭、一氧化碳、氢和活泼金属等都是良好的还原剂。


(1)炭热还原法:反应需要高温,常在高炉和电炉中进行。所以这种冶炼金属的方法又称为火法冶金。

氧化物矿:$\ce{SnO2 + 2C ->[{\mbox{高温}}] Sn + 2CO2 ^}$,$\ce{MgO + C ->[{\mbox{高温}}] Mg + CO ^}$

碳酸盐矿:一般重金属的碳酸盐受热时都能分解为氧化物,再用焦炭还原。

硫化物矿:先在空气中锻烧,使它变成氧化物,再用焦炭还原。如从方铅矿提取铅:$\ce{2PbS + 3O2 ->[{\mbox{高温}}] 2PbO + SO2 ^}$,$\ce{PbO + C ->[{\mbox{高温}}] Pb + CO ^}$

(2)氢热还原法:工业上要制取不含炭的金属常用氢还原法。生成热较小的氧化物,例如,氧化铜、氧化铁等,容易被氢
还原成金属。而具有很大生成热的氧化物,例如,氧化铝、氧化镁等,基本上不能被氢还原成金属。用高
纯氢和纯的金属氧化物为原料,可以制得很纯的金属。

(3)金属热还原法:铝是最常用的还原剂即铝热法。

$\ce{Cr2O3 + 2Al ->[{\mbox{高温}}] Al2O3 + 2Cr}$ \quad $\Delta_\mathrm{r} G^\ominus_\mathrm{m} = -622.9 \mathrm{kJ \cdot mol^{-1}}$

铝容易和许多金属生成合金。可采用调节反应物配比来尽量使铝完全反应而不残留在生成的金属中。
钙、镁一般不和各种金属生成合金,因此可用作钛、锆、铪、钒、铌、钽等氧化物的还原剂。如用活泼金属还原金属卤化物来制备钛:

$\ce{TiCl4 + 4Na ->[{\mbox{高温}}] Ti + 4NaCl}$,$\ce{TiCl4 + 2Mg ->[{\mbox{高温}}] Ti + 2MgCl2}$

选择哪一种金属(常用 Na、Mg、Ca、Al)做还原剂,除$\Delta_\mathrm{r} G^\ominus_\mathrm{m}$来判断外还要注意下几方面情况;(1)还原
力强;(2)容易处理;(3)不和产品金属生成合金;(4)可以得到高纯度的金属;(5)其它产物容易和生成金属
分离;(6)成本尽可能低。

\subsection{电解法}

排在铝前面的几种活泼金属,不能用一般还原剂使它们从化合物中还原出来。这些金属用电解法制取
最适宜,电解是最强的氧化还原手段。电解法有水溶液电解和熔盐电解法两种。活泼的金属如铝、镁、钙、钠等用熔融化合物电解法制备。


\section{颜 色}

\section{鉴定与测定}

\section{重要化合物的制取}

\section{特殊性质}

\end{document}